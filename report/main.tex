\documentclass{article}
\usepackage[left=3cm,right=3cm,top=3cm,bottom=3cm]{geometry}
\usepackage[utf8]{inputenc}

\title{Benchmarking Farhads Shit}
\author{Max Blankestijn, Loes Dekker \& Rintse van de Vlassakker}
\date{February 2020}

\begin{document}

\maketitle

\section{Introduction}
% Concurrency & Reo
% Onderzoeksvraag
% Overzicht van verslag

\section{Materials}
\subsection{Treo Compiler}
\subsection{Reo\_rs Compiler}
% ?? \subsection{Perf}
\subsection{Hardware}

\section{Implementation}
\subsection{Protocol 1: Sequencer}
A sequencer is a protocol with $N$ producers. The producers can only put items in a set order. We did not attach a consumer to our sequencer, the data items simply get lost after a put.

\subsubsection{Reo}

\subsubsection{C++}

\subsection{Protocol 2: Repeater}
A repeater is a simple program with one producer and one consumer. 
Every item that is put is copied $k-1$ times.
The next item can only be put after the first item has been consumed $k$ times.
\subsubsection{Reo}

\subsubsection{C++} %wellicht een tikkeltje te uitgebreidt?
The C++ version of protocol 2 consists of a mutex $p$ for the producer and a mutex $c$ for the consumer as well as a counter. Data can only be put when $p$ can be acquired. After a succesfull put, the counter is set to $k$ and $c$ is released. This enables the get function to acquire $c$. For every get, the count is decreased by one. The get function releases $c$ after each successful get while the counter $> 0$, which enables another get. When the counter equals 0, $p$ is released, which enables a new put, while $c$ is not released, which prevents any new gets until a put is made.

\subsection{Protocol 3: Alternator}
An alternator has $N$ producers and 1 consumer. All the producers have to put their data items at the same time, meaning that $N-1$ producers have to wait for the slowest producer. The consumer gets the data items in a set order.

\subsubsection{Reo}

\subsubsection{C++}


\subsection{Protocol 4: Exclusive Router}

\subsubsection{Reo}

\subsubsection{C++}

\section{Experiments}

\section{Discussion}

\section{Conclusion and Further Research}

\newpage

\bibliographystyle{alpha}
\bibliography{bibliography}

\end{document}
