\section{Conclusion}
In this report the performance of programs generated by the Reo Compiler~\ref{reo} is compared to the performance of C programs that have the same functionality. Four different programs with a variable amount of either producers or consumers were written in both languages. Unfortunately the Reo Compiler was not able to compile every program, meaning that the experiments could not be completed. The results that were obtained from the successful experiment showed that the C implementation mostly outperforms Reo, except on some smaller producer or consumer sizes.

\section{Further Research}
For further research we suggest completing the tests we had planned when the compiler allows, as well as benchmarking additional protocols. The scope of our experiment was fairly limited, and when the compiler permits extensive testing more significant results can be obtained.\\\\
%
The Sequencer and Early Async Replicator were the protocols that were successfully compiled consistently. This allows for more extensive research on the short term.
There is a clear separation in the type of scaling of the different Sequencer implementations. Running more experiments at finer intervals could more accurately depict the scaling of both implementations and reveal exactly at what point C overtakes Reo. In addition to finer intervals, more experiments at greater sizes could reveal how well the scaling of both the Reo and C implementations holds up as the given size grows.

